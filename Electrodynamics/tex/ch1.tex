\chapter{Vector Analysis}

\section{Vector Algebra}

This section reviews the basics of vectors; for the purpose of these notes, I will assume the reader already has a strong basic understanding of vectors. I will simply include some concepts that may be less familiar to the average reader.

The \vocab{vector triple product} $\vec{a}\times (\vec{b}\times \vec{c})$ can be simplified by the so-called ``$bac-cab$ rule":
\[\vec{a}\times(\vec{b}\times\vec{c})=\vec{b}(\vec{a}\cdot\vec{c})-\vec{c}(\vec{a}\cdot\vec{b}).\]
Note that cross-products are not associative, so the order of parentheses matters. By repeated application of this fact, it is never necessary for an expression to contain more than one cross product in any term. For example,
\[(\vec{a}\times \vec{b})\cdot(\vec{c}\times\vec{d})=(\vec{b}\times(\vec{c}\times \vec{d}))\cdot \vec{a}=(\vec{c}(\vec{b}\cdot\vec{d})-\vec{d}(\vec{b}\cdot\vec{c}))\cdot\vec{a}=(\vec{a}\cdot\vec{c})(\vec{b}\cdot\vec{d})-(\vec{a}\cdot\vec{d})(\vec{b}\cdot\vec{c}),\]
where in the first equality we utilized the \vocab{scalar triple product} identity $\vec{a}\cdot(\vec{b}\times\vec{c})=\vec{b}\cdot(\vec{c}\times\vec{a})$.