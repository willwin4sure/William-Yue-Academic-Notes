\chapter{Vector Analysis}

\section{Vector Algebra}

\subsection{Vector Operations}

\subsection{Vector Algebra: Component Form}

These first two subsections reviews the basics of vectors; for the purpose of these notes, I will assume the reader already has a strong basic understanding of vectors, so they will be left empty.

\subsection{Triple Products}

The output of a cross product $\vec{b}\times\vec{c}$ is a vector, so it can be dotted or crossed with another vector $\vec{a}$ to form a triple product.

The \vocab{scalar triple product} $\vec{a}\cdot(\vec{b}\times\vec{c})$ gives the signed volume of the parallelepiped generated by $\vec{a},\vec{b},\vec{c}$. Therefore,
\begin{claim}
$\vec{a}\cdot(\vec{b}\times\vec{c})=\vec{b}\cdot(\vec{c}\times\vec{a})=\vec{c}\cdot(\vec{a}\times\vec{b})$.
\end{claim}
Notice that we preserved the cyclic order of the vectors in this case; if we flip the order (e.g. $\vec{a}\cdot(\vec{c}\times\vec{b})$), then we obtain the same value with opposite sign. In component form, we have
\[\vec{a}\cdot(\vec{b}\times\vec{c})=\det\begin{bmatrix}
a_x & a_y & a_z\\
b_x & b_y & b_z\\
c_x & c_y & c_z
\end{bmatrix},\]

The \vocab{vector triple product} $\vec{a}\times (\vec{b}\times \vec{c})$ can be simplified by the so-called ``$bac-cab$ rule":
\begin{claim}
$\vec{a}\times(\vec{b}\times\vec{c})=\vec{b}(\vec{a}\cdot\vec{c})-\vec{c}(\vec{a}\cdot\vec{b}).$
\end{claim}
Note that cross-products are not associative, so the order of parentheses matters. By repeated application of this fact, it is never necessary for an expression to contain more than one cross product in any term. For example,
\[(\vec{a}\times \vec{b})\cdot(\vec{c}\times\vec{d})=(\vec{b}\times(\vec{c}\times \vec{d}))\cdot \vec{a}=(\vec{c}(\vec{b}\cdot\vec{d})-\vec{d}(\vec{b}\cdot\vec{c}))\cdot\vec{a}=(\vec{a}\cdot\vec{c})(\vec{b}\cdot\vec{d})-(\vec{a}\cdot\vec{d})(\vec{b}\cdot\vec{c}),\]
where in the first equality we utilized the scalar triple product identity $\vec{a}\cdot(\vec{b}\times\vec{c})=\vec{b}\cdot(\vec{c}\times\vec{a})$.

\subsection{Position, Displacement, and Separation Vectors}

\begin{definition}
Given a coordinate system with origin $\mathcal{O}$, the vector from $\mathcal{O}$ to some point is the \vocab{position vector}
\[\vec{r}:=x\hat{x}+y\hat{y}+z\hat{z},\]
where $(x,y,z)$ are the Cartesian coordinates of the point. 
\end{definition}
Note that the distance from the point to the origin is
\[r=|\vec{r}|=\sqrt{x^2+y^2+z^2}\]
and the unit vector in the direction of $\vec{r}$ is 
\[\hat{r}=\frac{\vec{r}}{r}=\frac{x\hat{x}+y\hat{y}+\hat{z}}{\sqrt{x^2+y^2+z^2}}.\]

\begin{definition}
The \vocab{infinitesimal displacement vector} from $(x,y,z)$ to $(x+dx,y+dy,z+dz)$ is denoted by
\[d\vec{\ell}=dx\hat{x}+dy\hat{y}+dz\hat{z}.\]
\end{definition}
Oftentimes in electrodynamics, we encounter problems involving two points:
\begin{itemize}
    \item A \vocab{source point} $\vec{r'}$, where an electric charge is located,
    \item A \vocab{field point} $\vec{r}$, where we are calculating the electric or magnetic field.
\end{itemize}
In the situation when $\vec{r'}\neq \vec{0}$, it is useful to adopt a shorthand notation for the \vocab{separation vector} from the source point to the field point; we will use the script letter
\[\vec{\scripty{r}}:=\vec{r}-\vec{r'}.\]

\subsection{How Vectors Transform}

What exactly is a vector? You might have learned it as ``a quantity with a magnitude and direction," but this isn't exactly satisfactory...

Perhaps a 3-dimensional vector is simply anything with three real components that combine properly under addition. However, what if we considered some silly example like a barrel with $N_x$ pear, $N_y$ apples, and $N_z$ bananas? Is $\vec{N}=N_x\hat{x}+N_y\hat{y}+N_z\hat{z}$ suddenly a vector? Surely not---what is the direction? What is wrong with it?

The answer is that $\vec{N}$ \textit{doesn't transform properly when you change coordinates}.

In three dimensions, if a $x,y,z$ system is rotated to some $\ol{x},\ol{y},\ol{z}$, system, then the coordinates of some vector $\vec{a}$ undergo a matrix transformation
\[
\begin{bmatrix}
\ol{a_x}\\\ol{a_y}\\\ol{a_z}
\end{bmatrix}=\begin{bmatrix}
R_{xx} & R_{xy} & R_{xz}\\
R_{yx} & R_{yy} & R_{yz}\\
R_{zx} & R_{zy} & R_{zz}
\end{bmatrix}\cdot\begin{bmatrix}
a_x\\a_y\\a_z
\end{bmatrix},
\]
where $R_{ij}$ are some products of trigonometric functions (or it could simply be any arbitrary linear transformation of $\mathbb{R}^3$). More compactly, we can write it as



\section{Differential Calculus}

\section{Integral Calculus}
