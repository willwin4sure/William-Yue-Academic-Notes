\chapter{Electrostatics}

\section{The Electric Field}

\subsection{Introduction}

The fundamental problem in electrodynamics is how \vocab{source charges} $q_1,q_2,\dots$ (whose positions are \textit{given} as functions of time) affect a \vocab{test charge} $Q$ (whose trajectory is to be \textit{calculated}).

The \vocab{law of superposition} helps: the net force $\vec{F}$ on $Q$ is just the vector sum of the forces $\vec{F}_i$ from $q_i$ on $Q$:
\[\vec{F}=\sum_i \vec{F}_i.\]

Sadly, each individual force $\vec{F}_i$ is not easy to calculate, either. Indeed, it depends not only on the separation distance $\scr$, but also \textit{both} their velocities and on the \textit{acceleration} of $q$. Further, electromagnetic ``news" only propagates at the speed of light, so we need to consider \textit{not} what each $q_i$ is doing now, but what it was doing at some slightly earlier time.

So we need to simplify the problem. In \vocab{electrostatics}---as the name suggests---we take all the source charges $q_i$ to be static, or stationary (the test charge $Q$ can still move).

\subsection{Coulomb's Law}

Recall:

\begin{theorem}[Coulomb's Law]
The force of a stationary charge $q$ on a test charge $Q$ that is $\scr$ away is given by
\[\vec{F}=\frac{1}{4\pi\varepsilon_0}\frac{qQ}{\scr^2}{\hat{\scr}}.\]
This is confirmed by experiments. 
\end{theorem}

\begin{definition}
The constant here is
\[\varepsilon_0=8.85\times 10^{-12}\frac{\text{C}^2}{\text{N}\cdot\text{m}^2},\]
called the \vocab{permittivity of free space}.
\end{definition}

Notice that the force $\vec{F}$ is repulsive if $q$ and $Q$ have the same sign, and attractive if they have different signs. Indeed, $\vec{\scr}$ points from $q$ to $Q$.

\subsection{The Electric Field}

Now suppose we have $n$ source charges $q_1,q_2,\dots,q_n$, with respective vectors $\vec{\scr_1},\vec{\scr_2},\dots,\vec{\scr_n}$ to a test charge $Q$. Then, by the superposition principle, we have that
\[\vec{F}=\sum_{i=1}^n\vec{F}_i=\frac{Q}{4\pi\varepsilon_0}\sum_{i=1}^n\frac{q}{\scr_i^2}\hat{\scr_i}.\]
Notice how this quantity is proportional to $Q$, so we can define another vector field that depends \textit{only} on the source charges:
\begin{definition}
We define the \vocab{electric field} generated by source charges $q_1,\dots,q_n$ at a point $\vec{r}$ as
\[\vec{E}(\vec{r}):=\frac{1}{4\pi\varepsilon_0}\sum_{i=1}^n\frac{q_i}{\scr_i^2}\hat{\scr_i}.\]
\end{definition}
Then, the force of a test charge $Q$ placed at $\vec{r}$ is $\vec{F}=Q\vec{E}$, by the definition of $\vec{E}$. Further, since $\vec{E}$ is independent of any test charges $Q$, we can imagine it as an actual ``real" physical field that permeates space, and when test charges $Q$ are placed in the field they get pushed along (or against) the electric field.

\subsection{Continuous Charge Distribution}\label{contchardist}

We have only covered the case where the source is a collection of discrete point charges: to cover the continuous case, we need to turn our sum into an integral:
\[\vec{E}(\vec{r})=\frac{1}{4\pi\varepsilon_0}\int \frac{dq}{\scr^2}\hat{\scr}.\]
\begin{itemize}
    \item If the charge is spread over a line, with linear charge density $\lambda$, then $dq=\lambda d\ell'$, so
    \[\vec{E}(\vec{r})=\frac{1}{4\pi\varepsilon_0}
    \int\frac{\lambda(\vec{r'})}{\scr^2}\hat{\scr}d\ell'.\]
    \item If the charge is spread over a plane, with surface charge density $\sigma$, then $dq=\sigma da'$, so
    \[\vec{E}(\vec{r})=\frac{1}{4\pi\varepsilon_0}\int\frac{\sigma(\vec{r'})}{\scr^2}\hat{\scr}da'.\]
    \item If the charge is spread over a volume, with volume charge density $\rho$, then $dq=\rho d\tau'$, so
    \[\boxed{\vec{E}(\vec{r})=\frac{1}{4\pi\varepsilon_0}\int\frac{\rho(\vec{r'})}{\scr^2}\hat{\scr}d\tau'}.\]
\end{itemize}

Often, the final boxed equation is referred to as ``Coulomb's Law": it is not far from our original formulation, and volume charge distributions are common.

\section{Divergence and Curl of Electrostatic Fields}

\subsection{Field Lines, Flux, and Gauss' Law}

In principle, we are done with electrostatics: just compute the boxed integral above to find the electric field generated by the source charges; we know that a test charge $Q$ will simply receive a force $\vec{F}=Q\vec{E}$ at a point with electric field $\vec{E}$.

However, that integral is often intractable, so much of electrostatics is actually concerned with \textbf{assembling a bag of tools and tricks} to avoid them.

One tool to get a feel for electric fields is through \vocab{field lines}. We consider the case of a single point charge $q$ at the origin; shown below are (a) some vectors in the electric field and (b) electric field lines.

\begin{center}
    \includegraphics[width=12cm]{Electrodynamics/images/fig2.12.PNG}
\end{center}

It may seem like we lost some information: the strength of the field. However, we actually haven't! It turns out the magnitude of the field is indicated by the \textbf{density} of the field lines. However, the picture is actually a bit deceptive when drawn in two dimensions: it seems like the field drops with $1/r$; however, in three dimensions, we do get the correct $1/r^2$.

Note that charges must have a number of field lines emanating/terminating at it proportional to the magnitude of the charge. Further, electric field lines cannot terminate at any points other than charges (else $\nabla\cdot\vec{E}=0$, something we'll prove later, would be violated): they must start/end at charges or at infinity. Further, field lines cannot cross (otherwise the electric field at that point would have two directions!). Depicted below are the field lines for an electric dipole.

\begin{center}
    \includegraphics[width=10cm]{Electrodynamics/images/fig2.13.PNG}
\end{center}

Further, using this model, there's a simple interpretation for the \textit{flux} of $\vec{E}$ through a surface $\mathcal{S}$:
\[\Phi_E=\iint_{\mathcal{S}}\vec{E}\cdot d\vec{a}\]
is simply the (signed) number of electric field lines passing through $\mathcal{S}$.

This suggests that the flux through a \textit{closed} surface should measure the total charge inside the surface; in particular, charges outside should have no affect on the flux since any field lines that enter simply exit again.

This is the essence of \vocab{Gauss's law}.

\begin{theorem}[Gauss's Law]
    For any closed surface $\mathcal{S}$, we have
    \[\oiint_{\mathcal{S}}\vec{E}\cdot d\vec{a}=\frac{Q_{\text{enc}}}{\varepsilon_0},\]
    where $Q_{\text{enc}}$ is the total charge enclosed within $\mathcal{S}$.
\end{theorem}

Note that since Newton's law of gravitation also obeys $1/r^2$ fall-off, it will also obey an analogous Gauss's law.

To turn this into differential form, we can rewrite the left hand side using the divergence theorem:
\[\oiint_{\mathcal{S}}\vec{E}\cdot d\vec{a}=\iiint_{\mathcal{V}}(\nabla\cdot \vec{E})d\tau\]
where $\partial\mathcal{V}=\mathcal{S}$; i.e. $\mathcal{V}$ is the interior of $\mathcal{S}$. In addition, we can write
\[\frac{Q_{\text{enc}}}{\varepsilon_0}=\iiint_{\mathcal{V}}\rho d\tau.\]
Hence.
\[\iiint_{\mathcal{V}}(\nabla\cdot\vec{E})d\tau=\iiint_{\mathcal{V}}\left(\frac{\rho}{\varepsilon_0}\right)d\tau.\]
However, since this holds true for \textit{any} volume of choice $\mathcal{V}$, the integrands must also be equal.

\begin{theorem}[Gauss's Law in Differential Form]
    We have that
    \[\nabla\cdot\vec{E}=\frac{\rho}{\varepsilon_0}.\]
\end{theorem}

\subsection{The Divergence of $\vec{E}$}

Let's calculate the divergence of $\vec{E}$ directly from the boxed equation in Section \ref{contchardist}:
\[\vec{E}(\vec{r})=\frac{1}{4\pi\varepsilon_0}\iiint_{\mathbb{R}^3}\frac{\hat{\scr}}{\scr^2}\rho(\vec{r'})d\tau'.\]
Since the only dependence on $\vec{r}$ is in $\vec{\scr}=\vec{r}-\vec{r'}$, and the divergence is linear, we can write
\[\nabla\cdot\vec{E}=\frac{1}{4\pi\varepsilon_0}\iiint_{\mathbb{R}^3}\nabla\cdot\left(\frac{\hat{\scr}}{\scr^2}\right)\rho(\vec{r'})d\tau'.\]
Recall that in Section \ref{3ddirac} we showed that $\nabla\cdot\left(\frac{\hat{\scr}}{\scr^2}\right)=4\pi\delta^3(\vec{\scr})$,
so
\[\nabla\cdot\vec{E}=\frac{1}{4\pi\varepsilon_0}\iiint_{\mathbb{R}^3}4\pi\delta^3(\vec{r}-\vec{r'})\rho(\vec{r'})d\tau'=\frac{\rho(\vec{r})}{\varepsilon_0},\]
as desired.

We can recover the original form of Gauss's law by simply integrating over a volume $\mathcal{V}$.

\subsection{Applications of Gauss's Law}

Gauss's law is \textbf{extraordinarily powerful}, especially when symmetry permits. 

\begin{definition}
When we apply Gauss's law over a closed surface $\mathcal{S}$ that the boundary of some volume, we generally refer to that surface as a \vocab{Gaussian surface}.
\end{definition}

\begin{example}[Electric Field of Ball]
Find the field outside a uniformly charged solid ball of radius $R$ and total charge $q$.
\end{example}

\begin{proof}
Suppose we wish to find the field $r$ away from the center: then, draw our Gaussian surface $\mathcal{S}$ as a sphere of radius $r$ that is concentric with our solid charged ball.

Gauss's law tells us that 
\[\oiint_{\mathcal{S}}\vec{E}\cdot d\vec{a}=\frac{Q_{enc}}{\varepsilon_0}=\frac{q}{\varepsilon_0}.\]
Now, by spherical symmetry, the electric fields $\vec{E}$ along $\mathcal{S}$ must all point directly outward (parallel to $d\vec{a}$), and have equal magnitude; hence,
\[E\cdot 4\pi r^2=\frac{q}{\varepsilon_0}\implies \vec{E}=\boxed{\frac{1}{4\pi \varepsilon_0}\frac{q}{r^2}\hat{r}}.\]
\end{proof}

This is actually the exact same as the electric field generated by a point charge $q$ placed at the center of the sphere, which is true by the shell theorem. 

In general, there are two types of symmetry: spherical symmetry and cylindrical symmetry. In the first case, make your Gaussian surface a concentric sphere; in the second, make your Gaussian surface a coaxial cylinder.

\begin{exercise}
A long cylinder carries a charge density that is proportional to the distance from the axis: $\rho=ks$, for some constant $k$. Find the electric field inside this cylinder.
\end{exercise}

\begin{exercise}[Electric Field of Plane]
An infinite plane carries a uniform surface charge $\sigma$. Find its electric field.
\end{exercise}

\begin{exercise}[Parallel Plate Capacitor]
Two infinite parallel planes carry equal but opposite uniform charge densities $\pm \sigma$. Find the field in each of the three regions: (i) to the left of both, (ii) between them, (iii) to the right of both.
\end{exercise}

\subsection{The Curl of $\vec{E}$}

We can calculate the curl of $\vec{E}$ in the simple case of a point charge at the origin:
\[\vec{E}=\frac{1}{4\pi\varepsilon_0}\frac{q}{r^2}\hat{r}.\]
If we calculate the line integral 
\[\int_{\vec{a}}^{\vec{b}}\vec{E}\cdot d\vec{\ell}\]
of this field from point $\vec{a}$ to point $\vec{b}$, remember that we can write
\[d\vec{\ell}=dr\hat{r}+rd\theta\hat{\theta}+r\sin\theta d\varphi \hat{\varphi}.\]
Hence,
\[\vec{E}\cdot d\vec{\ell}=\frac{1}{4\pi\varepsilon_0}\frac{q}{r^2}dr.\]
Hence
\[\int_{\vec{a}}^{\vec{b}}\vec{E}\cdot d\vec{\ell}=\frac{1}{4\pi\varepsilon_0}\int_{\vec{a}}^{\vec{b}}\frac{q}{r^2}dr=\frac{1}{4\pi\varepsilon_0}\left(\frac{q}{r_a}-\frac{q}{r_b}\right),\]
where $r_a,r_b$ are the distances of $\vec{a},\vec{b}$ from the origin, respectively.

Therefore, the line integral between two points $\vec{a}$ and $\vec{b}$ doesn't depend on the particular path chosen; so the loop integral of $\vec{E}$ is also equal to zero:
\[\oint \vec{E}\cdot d\vec{\ell}=0\]
for any closed loop. Therefore, by Stokes' Theorem, we have that $\nabla\times\vec{E}=\vec{0}$.

Since $\nabla$ is linear, this also applies for any set of discrete point charges by the superposition principle.

\section{Electric Potential}

\subsection{Introduction to Potential}

Electric fields aren't just any old vector fields. They are in a very special class of vector functions: ones that have no curl anywhere, i.e. $\nabla\times\vec{E}=\vec{0}$. However, becuase of this, we can use Theorem \ref{irrotfields} to reduce the \textit{vector} problem of finding $\vec{E}$ to the \textit{scalar} problem.

\begin{definition}
Since $\nabla\times\vec{E}=\vec{0}$, there exists some scalar field $V$ such that $\vec{E}=-\nabla V$. We call $V$ the \vocab{electric potential}. 
\end{definition}

If we agree on some standard reference point $\mathcal{O}$ in the plane, then it turns out the function
\[V(\vec{r}):=-\int_{\mathcal{O}}^{\vec{r}} \vec{E}(\vec{r'})\cdot d\vec{\ell'}.\]
Since $\vec{E}$ is path-independent, this integral depends only on $\vec{r}$. In particular, the potential \textit{difference} between any two points $\vec{a}$ and $\vec{b}$ is simply
\[V(\vec{b})-V(\vec{a})=-\int_{\vec{a}}^{\vec{b}}\vec{E}\cdot d\vec{\ell}.\]
This is equivalent to $\vec{E}=-\nabla V$ by the fundamental theorem of gradients:
\[V(\vec{b})-V(\vec{a})=\int_{\vec{a}}^{\vec{b}}(\nabla V)\cdot d\vec{\ell}.\]

\subsection{Comments on Potential}

\paragraph{The name.} The name ``potential" invariably reminds you of ``potential energy." There is a connection, but they are completely different terms that should have different names. A surface over which the potential is constant is called an \vocab{equipotential}.

\paragraph{Advantage of the potential formulation.} If you know $V$, you can easily get $\vec{E}$ just by taking the gradient:
\[\vec{E}=-\nabla V.\]
Yet $\vec{E}$ is a \textit{vector} quantity while $V$ is a \textit{scalar} quantity. How can \textit{one} function $V$ contain all the information for the \textit{three} components of $\vec{E}$.

The reason, as it turns out, is that $\vec{E}$ is a very special type of function, being irrotational. Indeed, the condition $\nabla\times \vec{E}$, in components, translates to
\[\frac{\partial E_x}{\partial y}=\frac{\partial E_y}{\partial x}, \qquad \frac{\partial E_z}{\partial y}=\frac{\partial E_y}{\partial z}, \qquad \frac{\partial E_x}{\partial z}=\frac{\partial E_z}{\partial x}.\]

\paragraph{The reference point.} There is an ambiguity in defining $V$, based on the choice of reference point $\mathcal{O}$. However, altering the reference point amounts to just shifting by a constant. 

In this sense, potential is a lot like altitude: if I asked you the height of something, you'd probably give it to me in terms of its height above sea level, but it would be just as valid to measure with respect to New York, or Greenwich, or wherever. Shifting by a constant doesn't matter; rather, the only quantity that actually matters is the \textit{difference} in altitude between any two points.

Nonetheless, there is a natural spot for $\mathcal{O}$ in electrostatics: and that is a point infinitely far from any charge. So, we set the zero of potential at infinity. Once again, this convention fails if the charge itself extends to infinity; in these cases, the potential often blows up. The remedy is to pick some other point in space as the reference point. Also note that in real life, we never have to deal with this issue.

\paragraph{Potential obeys the superposition principle.} Since the electric field obeys this principle and $\nabla$ is linear, so should the potential. However, it's even better now since it's an \textit{ordinary} sum of scalars, not vectors.

\paragraph{Units of potential.} As electric fields are newtons per coulomb, potential is newton-meters per coulomb, or joules per coulomb. We call this unit a \vocab{volt}.

\begin{example}
Find the potential inside and outside a spherical shell of radius $R$ that carries a uniform surface charge, and total charge $q$.
\end{example}

\begin{proof}[Solution]
    Recall that the electric field generated by the sphere is
    \[\vec{E}(\vec{r})=\begin{cases}
        \frac{1}{4\pi \varepsilon_0}\frac{q}{r^2}\hat{r} & \text{if }r>R,\\
        0 & \text{otherwise}.
    \end{cases}\]
    For points outside of the sphere we integrate
    \[V(r)=-\int_\infty^{\vec{r}}\vec{E}\cdot d\vec{\ell}=-\frac{1}{4\pi \varepsilon_0}\int_\infty^r \frac{q}{r'^2}dr'=\frac{1}{4\pi\varepsilon_0}\frac{q}{r} \qquad (r>R).\]
    For points inside the sphere, note that $\nabla V=\vec{0}$ inside, so we simply have 
    \[V(r)=\frac{1}{4\pi\varepsilon_0}\frac{q}{R}\qquad (r<R).\]
\end{proof}

\subsection{Poisson's Equation and Laplace's Equation}

Recall that we can write $\vec{E}=-\nabla V$. However, also recall that we've determined that
\[\nabla\cdot \vec{E}=\frac{\rho}{\varepsilon_0}\qquad \text{and}\qquad\nabla\times\vec{E}=\vec{0}.\]
Substituting in $V$ gives
\[\nabla\cdot (\nabla V)=-\frac{\rho}{\varepsilon_0}.\]
Recall that the divergence of the gradient is the \textit{Laplacian}. This gives \vocab{Poisson's Equation}.

\begin{theorem}[Poisson's Equation]
We have
\[\nabla^2 V=-\frac{\rho}{\varepsilon_0}.\]
\end{theorem}

In regions with no charge, this reduces to \vocab{Laplace's Equation},
\[\nabla^2 V=0.\]

What about the fact that $\vec{E}$ is irrotational? Sadly, it doesn't give anything, since the curl of a gradient (of $V$) should always equal $\vec{0}$ anyway.

\subsection{The Potential of a Localized Charge Distribution}

Usually, we want to find $\vec{E}$ from $V$, and $V$ from the charge distribution $\rho$. Sadly, Poisson's Equation is the wrong way around: we can take the Laplacian of $V$ to get $\rho$. Is there some way we can ``invert" Poisson's equation?

Recall that a point charge $q$ at the origin generates a potential
\[V(r)=\frac{1}{4\pi\varepsilon_0}\frac{q}{r}.\]
It turns out that we can generalize this to arbitrary charge distributions:
\[V(\vec{r})=\frac{1}{4\pi\varepsilon_0}\int\frac{1}{\scr}dq.\]
For volume charge distributions, $dq=\rho\cdot d\tau'$, so
\[\boxed{V(\vec{r})=\frac{1}{4\pi\varepsilon_0}\iiint\frac{\rho(\vec{r'})}{\scr}d\tau'}.\]
This is a solution to Poisson's equation for a localized charge distribution. Similarly, for surface and line charge distributions we get potentials
\[V=\frac{1}{4\pi\varepsilon_0}\int\frac{\lambda(\vec{r'})}{\scr}d\ell'\qquad\text{and}\qquad V=\frac{1}{4\pi\varepsilon_0}\iint\frac{\sigma(\vec{r'})}{\scr}da'.\]
Recall once again that this section depends on the fact that the reference point is at infinity.

\subsection{Boundary Conditions}

We have covered the three fundamental quantities in electrostatics: $\rho$, $\vec{E}$, and $V$. We have also derived all six formulas interrelating them, summarized below. All six of these formulas followed from two experimental observations: (1) the principle of superposition and (2) Coulomb's law.

\begin{center}
    \includegraphics[width=10cm]{Electrodynamics/images/fig2.35.PNG}
\end{center}

Now onto the point of this section: what happens to $\vec{E}$ and $V$ when you cross a \textit{boundary}, i.e. a surface with charge density $\sigma$. 

It is simple to determine this for $\vec{E}$ using Gauss's law: simply draw a really thin wafer of area $A$ and thickness $\epsilon$ containing the surface; then
\[\oiint_{\mathcal{S}}\vec{E}\cdot d\vec{a}=\frac{Q_{\text{enc}}}{\varepsilon_0}=\frac{\sigma A}{\varepsilon_0}.\]
Even if there are other electric charges in the area, the flux through the sides of the pillbox is 0 as we take $\epsilon\to 0$. Hence, the difference in the normal component of the electric field below and the \textit{normal} component (obtained after dotting with $d\vec{a}$) of the electric field above is $\frac{\sigma}{\varepsilon_0}$. Note that this is true for a single infinitely flat plane with surface charge density $\sigma$ as well: the electric fields are both of magnitude $\frac{\sigma}{2\varepsilon_0}$, but pointing in opposite directions.

However, the \textit{tangential} component of $\vec{E}$ is \textit{always} continuous. After all, the loop integral of $\vec{E}$ is always $0$, so if we considered a small loop that is the along the side of the pillbox (so length $\ell$ parallel to the surface and length $\epsilon$ through the surface), then we see that the tangential components must be equal (since their difference is 0).

Hence, we can summarize:
\[\vec{E}_{\text{above}}-\vec{E}_{\text{below}}=\frac{\sigma}{\varepsilon_0}\hat{n},\]
where $\hat{n}$ is a unit vector pointing from below to above.

On the other hand, the potential is continuous across \textit{any} boundary, since the difference in $V$ is given by the path integral of $\vec{E}$, but since the path length shrinks to zero, so too does the integral:
\[V_{\text{above}}-V_{\text{below}}=-\int_{\text{below}}^{\text{above}}\vec{E}\cdot d\vec{\ell}=0.\]

The \textit{gradient} of $V$ keeps the discontinuity though, so
\[\nabla V_{\text{above}}-\nabla V_{\text{below}}=-\frac{\sigma}{\varepsilon_0}\hat{n}.\]
More conveniently, we can consider the $\hat{n}$ component of both sides by taking the dot product with $\hat{n}$, recalling that the dot product of the gradient of $V$ with a unit vector gives the rate of change of $V$ along the direction of that unit vector:
\[\frac{\partial V}{\partial n}=(\nabla V)\cdot\hat{n};\]
in this case, we call it the \vocab{normal derivative} of $V$, i.e. the rate of change of $V$ in the direction perpendicular (normal) to the surface. Hence, we can write
\[\frac{\partial V_{\text{above}}}{\partial n}-\frac{\partial V_{\text{below}}}{\partial n}=-\frac{\sigma}{\varepsilon_0}.\]

\section{Work and Energy in Electrostatics}

\subsection{The Work It Takes to Move a Charge}

Suppose we have a \textit{stationary} configuration of source charges, and want to move a test charge $Q$ from $\vec{a}$ to $\vec{b}$. How much work will we have to do?

Well, the force at any point is given by $\vec{F}=Q\vec{E}$; so you need to exert $-Q\vec{E}$. Hence the needed work is
\[W=\int_{\vec{a}}^{\vec{b}}\vec{F}\cdot d\vec{\ell}=-Q\int_{\vec{a}}^{\vec{b}}\vec{E}\cdot d\vec{\ell}=Q[V(\vec{b})-V(\vec{a})].\]
As $\nabla\times \vec{E}=\vec{0}$, this answer is independent of the path from $\vec{a}$ to $\vec{b}$: therefore, we call the electrostatic force \textit{conservative}.

Hence,
\[V(\vec{b})-V(\vec{a})=\frac{W}{Q}.\]
This gives us another view of potential:
\begin{moral}
The potential difference between points $\vec{a}$ and $\vec{b}$ is equal to the work per unit charge required to carry a particle from $\vec{a}$ to $\vec{b}$.
\end{moral}
In particular, if you want to bring $Q$ in from far away at stick it at $\vec{r}$, you must do work $W=Q\cdot V(\vec{r})$ (as we set our reference point at infinity). In this sense, potential is \textit{potential energy per unit charge} of a single particle moving within a field generated by stationary source charges.

\subsection{The Energy of a Point Charge Distribution}

But what we want to assemble an entire \textit{collection} of point charges? How much work would we need?

The first charge $q_1$ takes zero work: there's nothing to help or fight against. Bringing in $q_2$ will cost $q_2V_1(\vec{r_2})$, where $V_1$ is the potential due to $q_1$, which is then evaluated at $\vec{r_2}$:
\[W_2=\frac{1}{4\pi \varepsilon_0}\frac{q_1q_2}{\scr_{12}},\]
where $\scr_{12}$ is the distance from $\vec{r_1}$ to $\vec{r_2}$. Moving in $q_3$ then takes
\[W_3=\frac{1}{4\pi\varepsilon_0}\frac{q_1q_3}{\scr_{13}}+\frac{1}{4\pi\varepsilon_0}\frac{q_2q_3}{\scr_{23}},\]
and so on. Continuing this pattern, the total work necessary is
\[W=\frac{1}{4\pi \varepsilon_0}\sum_{i=1}^n\sum_{j=i+1}^n\frac{q_iq_j}{\scr_{ij}}=\frac{1}{8\pi\varepsilon_0}\sum_{i=1}^n\sum_{j\neq i}^n\frac{q_iq_j}{\scr_{ij}}.\]
This is equal to
\begin{equation}\label{pointchargeenergy}
\frac{1}{2}\sum_{i=1}^nq_i\left(\sum_{j\neq i}^n\frac{1}{4\pi\varepsilon_0}\frac{q_i}{\scr_{ij}}\right)=\frac{1}{2}\sum_{i=1}^nq_iV(\vec{r_i}).
\end{equation}
This is exactly half of what you'd expect if you just added the work of placing any point charge into the configuration (holding everything else and their generated potential fixed).

This value represents the energy stored in the system of charges (avoiding the annoying term \textit{``potential" energy}).

\subsection{The Energy of a Continuous Charge Distribution}

For a volume charge density $\rho$, this becomes
\[W=\frac{1}{2}\iiint_{\mathbb{R}^3}\rho Vd\tau.\]
There is actually a wonderful way to rewrite this equation to get rid of $\rho$ and $V$ in place of $\vec{E}$. First, use Gauss's law to convert $\rho$ to $\vec{E}$:
\[\rho=\varepsilon_0
\nabla\cdot \vec{E}\implies W=\frac{\varepsilon_0}{2}\iiint_{\mathbb{R}^3}(\nabla\cdot \vec{E})Vd\tau.\]
Now, use integration by parts to transfer the derivative from $\vec{E}$ to $V$:
\[W=\frac{\varepsilon_0}{2}\left[-\iiint_{\mathbb{R}^3}E\cdot(\nabla V)d\tau+\oiint_{\partial \mathbb{R}^3}V\vec{E}\cdot d\vec{a}\right].\]
Wait a second... what's $\partial \RR^3$? Well... if we expand the volume containing charge over and over then the surface integral will drop to $0$ since area grows with $r^2$ but $V$ falls off with $1/r$ and $\vec{E}$ falls off with $1/r^2$.

Now we can use $\nabla V=-\vec{E}$ to get
\begin{equation}\label{energyelectric}
W=\frac{\varepsilon_0}{2}\iiint_{\RR^3}E^2d\tau.
\end{equation}
The energy stored in an electric field per unit volume, then, is $\frac{\varepsilon_0E^2}{2}$.

\subsection{Comments on Electrostatic Energy}

\paragraph{A perplexing ``inconsistency".} The equation 
\[W=\iiint\frac{\varepsilon_0 E^2}{2}d\tau\]
seems to imply that the energy of a stationary charge distribution is \textit{always positive}. Yet the energy of two equal but opposite charges a distance $\scr$ apart is apparently negative quantity $-(1/4\pi\varepsilon_0)(q^2/\scr)$, according to Equation \ref{pointchargeenergy}.

What's going on? Well, \textit{both are correct}. The problem here is that the energy in $-(1/4\pi\varepsilon_0)(q^2/\scr)$ \textbf{doesn't take into account the energy necessary to \textit{create} the point charges} (it supposes the point charges are already given, we're just assembling them), while Equation \ref{energyelectric} does. 

This is good, in fact, as the energy of a point charge is actually \textit{infinite}. By Equation \ref{energyelectric},
\begin{align*}
W&=\frac{\varepsilon_0}{2}\iiint_{\mathbb{R}^3}\frac{1}{(4\pi\varepsilon_0)^2}\cdot \frac{q^2}{r^4}d\tau\\
&=\frac{q^2}{32\pi^2\varepsilon_0}\int_{\varphi=0}^{2\pi}\int_{\theta=0}^\pi\int_{r=0}^\infty \frac{1}{r^4}\cdot r^2\sin\theta drd\theta d\varphi.\\
&=\frac{q^2}{8\pi\varepsilon_0}\int_0^\infty \frac{1}{r^2}dr=\infty.
\end{align*}

So Equation \ref{energyelectric} is more \textit{complete} in giving the total energy stored in the charge configuration. Yet for point charges, as we've seen, we definitely shouldn't include this, so Equation \ref{pointchargeenergy} assumes the point charges are already made (e.g. electrons), and cannot be taken apart. So the energy inside them doesn't matter.

\paragraph{Where is the energy stored?} Is the energy stored in the charge, or in the field? Presently, this is an unanswerable question: we can compute \textit{what} the total energy is, and several ways to compute it, but worrying \textit{where} it is is irrelevant. In radiation theory and general relativity, it is useful to regard the energy as stored in the field with energy density $\frac{\varepsilon E^2}{2}$ per unit volume.

However, in electrostatics, we can just as well say that the energy is stored in the charge, with density $\frac{1}{2}\rho V$.

\paragraph{The superposition principle.} Because electrostatic energy is \textit{quadratic} in the fields, it does \textit{not} obey a superposition principle. The energy of a compound system is \textit{not} the sum of the energies of its parts considered equally, as there are cross terms:
\[W_{\text{tot}}=\frac{\varepsilon_0}2\iiint (\vec{E_1}+\vec{E_2})^2d\tau=W_1+W_2+\varepsilon_0\iiint \vec{E_1}\cdot\vec{E_2}d\tau.\]
For example, if you double the charge everywhere, you quadruple the total energy.


