\chapter{Electric Fields in Matter}

\section{Polarization}

\subsection{Dielectrics}

Most everyday objects belong to one of two large classes: \vocab{conductors} and \vocab{insulators} (or \vocab{dielectrics}).

\begin{definition}
In an \vocab{insulator}, all charges are attached to specific atoms or molecules. They're on a tight leash, and all they an do is move a little \textit{within} the atom or molecule.
\end{definition}

This is in contrast to conductors, where there is an ``unlimited" supply of charges that are free to move around throughout the material, without being attached to particular atoms or molecules.

The microscopic displacements within a dielectric material is not as dramatic as the rearrangement of charge in a conductor, but their effects account for the characteristic behavior of dielectric materials. External electric fields can distort the charge distribution of a dielectric atom or molecule in \textit{two} ways: \textit{stretching} and \textit{rotating}.

\subsection{Induced Dipoles}

What happens when a neutral atom is placed in an electric field $\vec{E}$? 

An initial guess is that nothing should happen: the atom is not charged, so the field has no effect on it. However, this is incorrect. Even though the atom \textit{as a whole} is electrically neutral, there is still a positively charged core (the nucleus) and negatively charged electron cloud surrounding it. These two regions are pushed and pulled by the field.

If the field is large enough, then this could be enough to pull the atom apart completely, ``ionizing" it into a conudctor. With less extreme fields, however, an equilibrium is established as the positive and negative charges attract one another, holding the atom together. 

This leaves the atom \vocab{polarized}, with plus charge shifted slightly in the direction of the field and the minus charge the other way. The atom now has a tiny dipole moment $\vec{p}$, which points in the \textit{same direction} as $\vec{E}$. Typically, these are roughly proportional
\[\vec{p}=\alpha\vec{E},\]
where the constant of proportionality is called the \vocab{atomic polarizability}.

\begin{example}\label{primatompolar}
A primitive model for an atom consists of a point nucleus $(+q)$ surrounded by a uniformly charged spherical cloud $(-q)$ of radius $a$. Calculate the atomic polarizability of such an atom.
\end{example}

\begin{proof}
Suppose the $+q$ nucleus is shifted a distance of $d$ in the direction of the electric field $\vec{E}$. Then, the field due to the spherical cloud pulling it back should be equal to $E$. Hence,
\[\frac{1}{4\pi\varepsilon_0}\cdot \frac{q\cdot d^3/a^3}{d^2}=\frac{qd}{4\pi\varepsilon_0 a^3},\]
so $p=4\pi\varepsilon_0 a^3\cdot E$. Hence,
\[\alpha=4\pi\varepsilon_0a^3=3\varepsilon_0 v,\]
where $v$ is the volume of the atom.
\end{proof}

\begin{remark}
Though this model is quite crude, the result isn't actually that bad---it's accurate within a factor of four or so for many simple atoms.
\end{remark}

Molecules are not as simple as atoms, since frequently they can polarize more easily in some directions than in others. For example, carbon dioxide has a polarizability of $4.5\times 10^{-40} \text{ C}^2\cdot\text{m}/\text{N}$ when a field is applied along the axis of the molecules, but only $2\times 10^{-40}$ for fields perpendicular to the axis. Therefore, if the field is at some angle to the axis, we resolve it into parallel and perpendicular components:
\[\vec{p}=\alpha_\perp\vec{E}_\perp + \alpha_\parallel \vec{E}_\parallel.\]
In these cases, the induced dipole moment may \textit{not} be in the same direction as $\vec{E}$.

For more general molecules (i.e. a completely asymmetrical molecule), we replace the equation with the most general linear relation between $\vec{E}$ and $\vec{p}$:
\[\begin{bmatrix}
p_x \\ p_y \\ p_z
\end{bmatrix}=\begin{bmatrix}
\alpha_{xx} & \alpha_{xy} & \alpha_{xz}\\
\alpha_{yx} & \alpha_{yy} & \alpha_{yz}\\
\alpha_{zx} & \alpha_{zy} & \alpha_{zz}
\end{bmatrix}\cdot\begin{bmatrix}
E_x \\ E_y \\ E_z
\end{bmatrix}.\]
The set of nine constants $\alpha_{ij}$ make up the \vocab{polarizability tensor} for the molecule. These values depend on the orientation of the coordinate axes that we choose, but note that we can always choose ``principal" axes so that the off-diagonal terms vanish (the matrix is diagonalizable).

\subsection{Alignment of Polar Molecules}

The neutral atom discussed in Example \ref{primatompolar} had no dipole moment to start with: $\vec{p}$ was induced by the external electric field $\vec{E}$. However, some molecules have built-in, permanent dipole moments. For example, in a water molecule, the electrons tend to cluster around the oxygen atom, and since it is bent at $105^\circ$, there is a negative charge built up at the vertex and a positive charge on the opposite side. The dipole moment thus generated is unusually large: $6.1\times 10^{-30}\text{ C}\cdot\text{m}$.

What happens when such \vocab{polar molecules} are placed in an electric field? Well, the \textit{force} on the positive end exactly cancels out the force on the negative end, but a torque is induced:
\begin{align*}
\vec{N}&=(\vec{r}_+\times \vec{F}_+)+(\vec{r}_-\times \vec{F}_-)\\
&=\left[(\vec{d}/2)\times (q\vec{E})\right]+\left[(-\vec{d}/2)\times (-q\vec{E})\right]=q\vec{d}\times \vec{E}.
\end{align*}
Thus a dipole $\vec{p}=q\vec{d}$ in a uniform field $\vec{E}$ experiences a torque
\[\vec{N}=\vec{p}\times\vec{E}.\]
Note that $\vec{N}$ is in a direction which tries to line $\vec{p}$ up with $\vec{E}$. Therefore, a polar molecule that is free to rotate will swing around until it points in the direction of the applied field.

If the field is \textit{nonuniform}, then there will be a net \textit{force} on the dipole, in addition to the torque. Normally this is not much of an issue, since $\vec{E}$ would have to change very abruptly in teh space of one molecule. Nonetheless, we have
\[\vec{F}=\vec{F}_++\vec{F}_-=q(\vec{E}_+-\vec{E}_-)=q(\Delta \vec{E}).\]
If the dipole is short, we can approximate the small change in $E_x$ along the direction $\vec{d}$ by
\[\Delta E_x=(\nabla E_x)\cdot \vec{d}.\]
This gives
\[\Delta\vec{E}=(\vec{d}\cdot \nabla)\vec{E},\]
so
\[\vec{F}=(\vec{p}\cdot \nabla)\vec{E}.\]

\subsection{Polarization}

So, what happens to a piece of dielectric material when it is placed in an electric field? From our discussion in the previous sections:\footnote{The actual process is not quite as clear-cut as these two bullets would suggest: polar molecules also experience some polarization by displacement (though generally rotation is much easier so that's the dominant effect), and some materials retain their polarization after the field is removed.}
\begin{itemize}
    \item If the substance consists of neutral atoms (or nonpolar molecules), the field will induce a tiny dipole moment pointing in the same direction as the field.
    \item If the substance consists of polar molecules, each permanent dipole will experience a torque, tending it to line up along the direction of the field (though random thermal motions will compete with this process).
\end{itemize}
The net result, either way, is that we end up with a lot of little dipoles pointing along the direction of the field, making the material \vocab{polarized}. A measure of this effect is the \vocab{polarization} $\vec{P}$, defined as the dipole moment per unit volume.

In the next section, we will study the field that a chunk of polarized material \textit{itself} produces. After that, we will put together the original field that caused $\vec{P}$ and the new field due to $\vec{P}$.

\section{The Field of a Polarized Object}

\subsection{Bound Charges}

The main question of this section is to find the field produced by an object with polarization $\vec{P}$ (i.e. dipole moment per unit volume) given.

We work with potentials. For a single dipole $\vec{p}$, recall that
\[V(\vec{r})=\frac{1}{4\pi\varepsilon_0}\frac{\vec{p}\cdot\hat{\scr}}{\scr^2},\]
where $\scr$ is the separation vector from the dipole to the point at which we are evaluating the potential, $\vec{r}$. Since we have a dipole moment $\vec{p}=\vec{P}d\tau'$ for each volume element $d\tau'$, we simply integrate for the total potential:
\[V(\vec{r})=\frac{1}{4\pi\varepsilon_0}\iiint_{\mathcal{V}}\frac{\vec{P}(\vec{r'})}{\scr^2}d\tau'.\]
In principle, we are done, but we can rearrange this into a more enlightening form by noting that
\[\nabla'\left(\frac{1}{\scr}\right)=\frac{\hat{\scr}}{\scr^2},\]
where the differentiation in $\nabla'$ is with respect to the \textit{source} coordinates in $\vec{r;}$. Hence, applying integration by parts gives
\begin{align*}
    V&=\frac{1}{4\pi\varepsilon_0}\iiint_{\mathcal{V}}\vec{P}\cdot \nabla'\left(\frac{1}{\scr}\right)d\tau'\\
    &=\frac{1}{4\pi\varepsilon_0}\left[\iiint_{\mathcal{V}}\nabla'\cdot\left(\frac{\vec{P}}{\scr}\right)d\tau'-\iiint_{\mathcal{V}}\frac{1}{\scr}(\nabla'\cdot \vec{P})d\tau'\right]\\
    &=\frac{1}{4\pi\varepsilon_0}\oiint_{\partial\mathcal{V}}\frac{\vec{P}}{\scr}\cdot d\vec{a'}-\frac{1}{4\pi\varepsilon_0}\iiint_{\mathcal{V}}\frac{1}{\scr}(\nabla'\cdot \vec{P})d\tau'.
\end{align*}
We define two quantities:
\begin{definition}
In a polarized dielectric, we define two \vocab{bound charges} $\sigma_b$ and $\rho_b$. The surface bound charge is given by
\[\sigma_b:=\vec{P}\cdot \hat{n},\]
while the volume bound charge is given by
\[\rho_b:=-\nabla\cdot\vec{P}.\]
\end{definition}

With these definitions, we simply have
\[V(\vec{r})=\frac{1}{4\pi\varepsilon_0}\oiint_{\partial\mathcal{V}}\frac{\sigma_b}{\scr}da'+\frac{1}{4\pi\varepsilon_0}\iiint_{\mathcal{V}}\frac{\rho_b}{\scr}d\tau'.\]
The first term here is simply the potential generated by a surface charge density $\sigma_b$, while the second term here is simply the potential generated by a volume charge density $\rho_b$.

\begin{remark}\label{boundchargeintuition}
In the next subsection, we will see how these bound charges correspond to actual accumulations of charge within the dielectric. I'd encourage you to think about why this would be true! (Think of a bunch of little electric dipoles.)
\end{remark}

Therefore, instead of integrating the contributions of all the individual infinitesimal dipoles, we could first calculate the bound charges, and then calculate the fields \textit{they} produce, using any standard techniques we use for surface or volume charges (e.g. Gauss's Law).

\begin{example}
Find the electric field produced by a uniformly polarized sphere of radius $R$.
\end{example}

\begin{proof}
Let the direction of polarization point along the positive $z$-axis. Since $\vec{P}$ is uniform, it is simple to compute the bound charges:
\[\sigma_b=\vec{P}\cdot \hat{n}=P\cos\theta \qquad \text{and} \qquad \rho_b=-\nabla\cdot\vec{P}=0.\]
Recall that in Example \ref{chargespherepotential} and the following remark, we computed the potential generated by such a spherical charge distribution to be
\[V(r,\theta)=\begin{dcases}
\frac{P}{3\varepsilon_0}r\cos\theta & \text{if } r\le R,\\
\frac{P}{3\varepsilon_0}r\left(\frac{R}{r}\right)^3\cos\theta & \text{if } r\ge R.
\end{dcases}\]
For the electric field inside the sphere, simply note that $r\cos\theta=z$, so the field inside is actually \textit{uniform}!
\[\vec{E}=-\nabla V=-\frac{P}{3\varepsilon_0}\hat{z}=-\frac{1}{3\varepsilon_0}\vec{P}\qquad (r<R).\]
Meanwhile, outside the sphere, the potential is actually just identical to a perfect dipole at the origin with dipole moment $\vec{p}=\frac{4}{3}\pi R^3\vec{P}$ (not very surprising, as this is the total dipole moment):
\[V=\frac{1}{4\pi\varepsilon_0}\frac{\vec{p}\cdot\hat{r}}{r^2}\qquad (r\ge R).\]
Here's a picture:
\begin{center}
    \includegraphics[width=8cm]{Electrodynamics/images/fig4.10.PNG}
\end{center}
\end{proof}

\subsection{Physical Interpretation of Bound Charges}

In the last section, we showed that instead of integrating over each of the tiny dipoles in a polarized dielectric material, we could simply compute some bound charge densities $\sigma_b$ and $\rho_b$ using $\vec{P}$ and then determine the field of these charges.

While we found the formulas $\sigma_b=P\cdot \hat{n}$ and $\rho_b=-\nabla\cdot \vec{P}$ using a mathematical trick, it turns out that there is a physical interpretation for these bound charges: they actually represent \textit{real, genuine accumulations of charge} within the dielectric material, as hinted at in Remark \ref{boundchargeintuition}.

The basic idea is very similar to the intuitive proof of Stokes' theorem: if we each little dielectric as a small $+$ next to a small $-$, we notice that these can often cancel each other. For a first example, imagine a string of dipoles lined up (with constant polarization), as shown below. The $+$ charge of one dipole will cancel with the $-$ charge of the next dipole, resulting in the only remaining charges being the bound $-$ on one end and bound $+$ on the other.

\begin{center}
    \includegraphics[width=8cm]{Electrodynamics/images/fig4.11.PNG}
\end{center}

To calculate the actual amount charge, consider a tube of dielectric parallel to $\vec{P}$. Consider the chunk of dielectric with dipole moment $P(Ad)$ shown on the left; it is equivalent to a dipole with charge $q=PA$. Hence the bound charge that piles up at the right end of the tube is $q=PA$.

Note that if the ends of the tube are indeed perpendicular to $\vec{P}$, then the surface charge density is simply $\sigma_b=P$, but if sliced diagonally, the \textit{total} charge should remain the same, but the new area is $A_{\text{end}}=\frac{A}{\cos\theta}$< so the surface charge density is instead $\vec{P}\cdot \hat{n}$ (which matches for the case where $\hat{n}$ is parallel to $\vec{P}$, anyway). 

\begin{center}
    \includegraphics[width=10cm]{Electrodynamics/tex/fig4.12-13.PNG}
\end{center}

Thus, we conclude that the effect of polarization of a dielectric is a bound charge of $\sigma_b=\vec{P}\cdot\hat{n}$ over the surface of the material.

So, where does the volume bound charge $\rho_b$ come from? Well, so far, we've only considered situations where $\vec{P}$ is uniform. If it's not, then we expect some accumulation of charge within the material. For example, if we have a strong area of polarization pointing into a weaker area of polarization, we expect an accumulation of positive charge within the material at that location. As for another example, consider the figure below:

\begin{center}
    \includegraphics[width=8cm]{Electrodynamics/images/fig4.14.PNG}
\end{center}

This immediately suggests that a diverging $\vec{P}$ results in an accumulation of negative charge. Indeed, if we consider the net bound charge in a particular volume $\mathcal{V}$, it should be equal and opposite to the amount of charge that is pushed out of the surface (which, as we've already reasoned, should be $\vec{P}\cdot\hat{n}$):
\[\iiint_{\mathcal{V}}\rho_bd\tau=-\oiint_{\partial\mathcal{V}}\vec{P}\cdot d\vec{a}=-\iiint_{\mathcal{V}}(\nabla\cdot \vec{P})d\tau.\]
Since this is true for \textit{any} volume, we must have $\rho_b=-\nabla\cdot \vec{P}$.

\begin{remark}
We can also interpret a uniformly polarized sphere as simply \textit{two} spheres of charge; a positive sphere and a negative sphere, superimposed. When the sphere is polarized, all the plus charges move slightly upwards and all the minus charges move slightly downwards (assuming polarization in the $\hat{z}$ direction). The leftover caps of charges is precisely the bound surface charge $\sigma_b$, while there is no charge accumulation $\rho_b$ inside since $\vec{P}$ is uniform.

For the field inside, you can easily compute it as uniform, and you'll get the same result of 
\[\vec{E}=-\frac{1}{3\varepsilon_0}\vec{P},\]
as before. For the field outside the sphere, we can simply imagine the charge on each sphere as concentrated in its center, so we get that the field outside is the same as a perfect dipole at the center (since the shift $\vec{d}$ between the centers of the negative and positive spheres is miniscule, at the scale of one atomic radius).
\end{remark}

% \subsection{The Field Inside a Dielectric (Optional)}

% In this section, we have mostly neglected any difference between physical and perfect dipoles; indeed, we started with pure dipoles making up the material. Yet, an actual polarized material consists of physical dipoles, even if they are small.

% Outside the material, this is fine. We are far away from these tiny dipoles, so the dipole potential is essentially correct. Yet what about \textit{inside} the material? We can hardly assume that that we are far away from the dipoles!

% Well, inside \textit{any} material, the actual \vocab{microscopic} electric field \textit{must} be fantastically complicated. After all, it must be massive in magnitude, say, really close to an electron, while just a short distance away it could be really small or in a completely different direction. Therefore, just as how we consider water to be a continuous fluid rather than truly being a bunch of microscopic particles, we only concentrate on the \vocab{macroscopic} electric field within a dielectric.

% To do so, we need to pick some characteristic radius (say, 1000 times the diameter of an atom) and average the electric field over it, enough to smooth over the microscopic bumps but also pick up on any large-scale variations (much like as we do for fluid mechanics when calculating pressure).

\section{The Electric Displacement}


