\chapter{Electric Fields in Matter}

\section{Polarization}

\subsection{Dielectrics}

Most everyday objects belong to one of two large classes: \vocab{conductors} and \vocab{insulators} (or \vocab{dielectrics}).

\begin{definition}
In an \vocab{insulator}, all charges are attached to specific atoms or molecules. They're on a tight leash, and all they an do is move a little \textit{within} the atom or molecule.
\end{definition}

This is in contrast to conductors, where there is an ``unlimited" supply of charges that are free to move around throughout the material, without being attached to particular atoms or molecules.

The microscopic displacements within a dielectric material is not as dramatic as the rearrangement of charge in a conductor, but their effects account for the characteristic behavior of dielectric materials. External electric fields can distort the charge distribution of a dielectric atom or molecule in \textit{two} ways: \textit{stretching} and \textit{rotating}.

\subsection{Induced Dipoles}

What happens when a neutral atom is placed in an electric field $\vec{E}$? 

An initial guess is that nothing should happen: the atom is not charged, so the field has no effect on it. However, this is incorrect. Even though the atom \textit{as a whole} is electrically neutral, there is still a positively charged core (the nucleus) and negatively charged electron cloud surrounding it. These two regions are pushed and pulled by the field.

If the field is large enough, then this could be enough to pull the atom apart completely, ``ionizing" it into a conudctor. With less extreme fields, however, an equilibrium is established as the positive and negative charges attract one another, holding the atom together. 

This leaves the atom \vocab{polarized}, with plus charge shifted slightly in the direction of the field and the minus charge the other way. The atom now has a tiny dipole moment $\vec{p}$, which points in the \textit{same direction} as $\vec{E}$. Typically, these are roughly proportional
\[\vec{p}=\alpha\vec{E},\]
where the constant of proportionality is called the \vocab{atomic polarizability}.

\begin{example}\label{primatompolar}
A primitive model for an atom consists of a point nucleus $(+q)$ surrounded by a uniformly charged spherical cloud $(-q)$ of radius $a$. Calculate the atomic polarizability of such an atom.
\end{example}

\begin{proof}
Suppose the $+q$ nucleus is shifted a distance of $d$ in the direction of the electric field $\vec{E}$. Then, the field due to the spherical cloud pulling it back should be equal to $E$. Hence,
\[\frac{1}{4\pi\varepsilon_0}\cdot \frac{q\cdot d^3/a^3}{d^2}=\frac{qd}{4\pi\varepsilon_0 a^3},\]
so $p=4\pi\varepsilon_0 a^3\cdot E$. Hence,
\[\alpha=4\pi\varepsilon_0a^3=3\varepsilon_0 v,\]
where $v$ is the volume of the atom.
\end{proof}

\begin{remark}
Though this model is quite crude, the result isn't actually that bad---it's accurate within a factor of four or so for many simple atoms.
\end{remark}

Molecules are not as simple as atoms, since frequently they can polarize more easily in some directions than in others. For example, carbon dioxide has a polarizability of $4.5\times 10^{-40} \text{ C}^2\cdot\text{m}/\text{N}$ when a field is applied along the axis of the molecules, but only $2\times 10^{-40}$ for fields perpendicular to the axis. Therefore, if the field is at some angle to the axis, we resolve it into parallel and perpendicular components:
\[\vec{p}=\alpha_\perp\vec{E}_\perp + \alpha_\parallel \vec{E}_\parallel.\]
In these cases, the induced dipole moment may \textit{not} be in the same direction as $\vec{E}$.

For more general molecules (i.e. a completely asymmetrical molecule), we replace the equation with the most general linear relation between $\vec{E}$ and $\vec{p}$:
\[\begin{bmatrix}
p_x \\ p_y \\ p_z
\end{bmatrix}=\begin{bmatrix}
\alpha_{xx} & \alpha_{xy} & \alpha_{xz}\\
\alpha_{yx} & \alpha_{yy} & \alpha_{yz}\\
\alpha_{zx} & \alpha_{zy} & \alpha_{zz}
\end{bmatrix}\cdot\begin{bmatrix}
E_x \\ E_y \\ E_z
\end{bmatrix}.\]
The set of nine constants $\alpha_{ij}$ make up the \vocab{polarizability tensor} for the molecule. These values depend on the orientation of the coordinate axes that we choose, but note that we can always choose ``principal" axes so that the off-diagonal terms vanish (the matrix is diagonalizable).

\subsection{Alignment of Polar Molecules}

The neutral atom discussed in Example \ref{primatompolar} had no dipole moment to start with: $\vec{p}$ was induced by the external electric field $\vec{E}$. However, some molecules have built-in, permanent dipole moments. For example, in a water molecule, the electrons tend to cluster around the oxygen atom, and since it is bent at $105^\circ$, there is a negative charge built up at the vertex and a positive charge on the opposite side. The dipole moment thus generated is unusually large: $6.1\times 10^{-30}\text{ C}\cdot\text{m}$.

What happens when such \vocab{polar molecules} are placed in an electric field? Well, the \textit{force} on the positive end exactly cancels out the force on the negative end, but a torque is induced:
\begin{align*}
\vec{N}&=(\vec{r}_+\times \vec{F}_+)+(\vec{r}_-\times \vec{F}_-)\\
&=\left[(\vec{d}/2)\times (q\vec{E})\right]+\left[(-\vec{d}/2)\times (-q\vec{E})\right]=q\vec{d}\times \vec{E}.
\end{align*}
Thus a dipole $\vec{p}=q\vec{d}$ in a uniform field $\vec{E}$ experiences a torque
\[\vec{N}=\vec{p}\times\vec{E}.\]
Note that $\vec{N}$ is in a direction which tries to line $\vec{p}$ up with $\vec{E}$. Therefore, a polar molecule that is free to rotate will swing around until it points in the direction of the applied field.

If the field is \textit{nonuniform}, then there will be a net \textit{force} on the dipole, in addition to the torque. Normally this is not much of an issue, since $\vec{E}$ would have to change very abruptly in teh space of one molecule. Nonetheless, we have
\[\vec{F}=\vec{F}_++\vec{F}_-=q(\vec{E}_+-\vec{E}_-)=q(\Delta \vec{E}).\]
If the dipole is short, we can approximate the small change in $E_x$ along the direction $\vec{d}$ by
\[\Delta E_x=(\nabla E_x)\cdot \vec{d}.\]
This gives
\[\Delta\vec{E}=(\vec{d}\cdot \nabla)\vec{E},\]
so
\[\vec{F}=(\vec{p}\cdot \nabla)\vec{E}.\]

\subsection{Polarization}

So, what happens to a piece of dielectric material when it is placed in an electric field? From our discussion in the previous sections:\footnote{The actual process is not quite as clear-cut as these two bullets would suggest: polar molecules also experience some polarization by displacement (though generally rotation is much easier so that's the dominant effect), and some materials retain their polarization after the field is removed.}
\begin{itemize}
    \item If the substance consists of neutral atoms (or nonpolar molecules), the field will induce a tiny dipole moment pointing in the same direction as the field.
    \item If the substance consists of polar molecules, each permanent dipole will experience a torque, tending it to line up along the direction of the field (though random thermal motions will compete with this process).
\end{itemize}
The net result, either way, is that we end up with a lot of little dipoles pointing along the direction of the field, making the material \vocab{polarized}. A measure of this effect is the \vocab{polarization} $\vec{P}$, defined as the dipole moment per unit volume.

In the next section, we will study the field that a chunk of polarized material \textit{itself} produces. After that, we will put together the original field that caused $\vec{P}$ and the new field due to $\vec{P}$.

