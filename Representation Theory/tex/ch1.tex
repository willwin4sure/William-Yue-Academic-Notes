\chapter{Introduction}

Representation theory involves studying algebra structures by representing their elements as linear maps between vector spaces. It has applications ranging from number theory and combinatorics to geometry, probability theory, quantum mechanics, and quantum field theory.

An important sub-problem is the representation theory of groups---since linear algebra is nicer and easier to work with than abstract algebra, it is nice to be able to take groups and represent them as sets of matrices instead.

As prerequisites to reading these notes, I suggest backgrounds in both abstract and linear algebra. 